\documentclass[
  french,
  a4paper
]{resume-openfont}

\begin{document}

\noindent%
\begin{minipage}{.5\textwidth}
\kern-0.35em
{
  \fontsize{28pt}{28pt}
  \fontspec[Path = fonts/roboto/]{Roboto-Light}\selectfont Antoine Bolvy\\
}
\textit{\myage{1996}{09}{04} ans, passionné d'informatique, solveur de problèmes.}\\
\textbf{Disponible pour CDI à partir de septembre 2018}
\vspace{5pt}


\end{minipage}% important to remove space between minipages, do not remove comment
\begin{minipage}{.5\textwidth}
\begin{flushright}
\color{headings}\fontspec[Path = fonts/raleway/]{Raleway-Regular}+33.6.75.45.53.49\\
\href{mailto:antoine.bolvy@gmail.com}{antoine.bolvy@gmail.com}\\
Paris, France\\
{\color{date}\fontsize{7pt}{12pt}\selectfont{}Dernière mise à jour: \today}
\end{flushright}
\end{minipage}

% \color{headingsrulegray}\hrule \linewidth 4pt
\noindent\makebox[\linewidth]{
{\color{headingsrulegray}\rule{\paperwidth}{0.4pt}}}
\vspace{-18pt}

%%%%%%%%%%%%%%%%%%%%%%%%%%%%%%%%%%%%%%
%
%     COLUMN ONE
%
%%%%%%%%%%%%%%%%%%%%%%%%%%%%%%%%%%%%%%

\noindent%
\begin{minipage}[t]{0.31\textwidth}


%%%%%%%%%%%%%%%%%%%%%%%%%%%%%%%%%%%%%%
%     EDUCATION
%%%%%%%%%%%%%%%%%%%%%%%%%%%%%%%%%%%%%%

\section{Éducation}

\subsection{Epitech}
\subtitle{Master en Informatique}
\location{2018 (Expecté) | Paris, France}
\sectionsep

\subsection{Epitech}
\subtitle{Bachelor en Informatique}
\location{2013 - 2016 | Lyon, France}
Bachelor GPA: 3.71/4.00
\vspace{0.8\topsep} % Hacky fix for awkward extra vertical space
\begin{coursework}
\item Artificial Intelligence
\item Computer Graphics
\item Networks TCP/IP
\item Linux Programming, System APIs
\item Object-Oriented Algorithms
\end{coursework}
\sectionsep

\subsection{Concordia University}
\subtitle{Étudiant international en informatique, niveau master}
\location{2016 - 2017 | Montréal, Canada}
\begin{coursework}
\item Multicore Programming
\item Computer Animation
\item Advanced Game Development
\item Operating Systems
\end{coursework}

%%%%%%%%%%%%%%%%%%%%%%%%%%%%%%%%%%%%%%
%     LINKS
%%%%%%%%%%%%%%%%%%%%%%%%%%%%%%%%%%%%%%

\section{Liens}
\begin{tabular}{@{}l@{\hskip 0.5em}l}
Github & \href{https://github.com/saveman71}{\bf /saveman71} \\
Linkedin & \href{https://www.linkedin.com/in/antoinebolvy}{\bf /in/antoinebolvy} \\
Twitter & \href{https://twitter.com/saveman71}{\bf @saveman71} \\
Facebook & \href{https://facebook.com/saveman71}{\bf /saveman71} \\
Site web & \href{https://saveman71.com}{\bf {\NoAutoSpacing https://saveman71.com}} \\
\end{tabular}
\sectionsep

%%%%%%%%%%%%%%%%%%%%%%%%%%%%%%%%%%%%%%
%     SKILLS
%%%%%%%%%%%%%%%%%%%%%%%%%%%%%%%%%%%%%%

\section{Compétences}
\subsection{Programmation}
\vspace{2pt}
\location{Utilisé pour de large projets:}
Java \textbullet{} Android \textbullet{} Python \textbullet{} Javascript \textbullet{} Node.js \textbullet{} C \textbullet{} C++ 11 and 14 \textbullet{} Shell \textbullet{} \altpdftext{\MyLaTeX}{LaTeX} \textbullet{} git \textbullet{} HTML5/CSS3\\
\vspace{2pt}%
\location{Utilisé pour des petits projets:}
Golang \textbullet{} CUDA/OpenCL \textbullet{} PHP\\
\vspace{2pt}%
\location{Bases:}
Ruby \textbullet{} C\# \textbullet{} MySQL
\sectionsep

\subsection{Langues}
\vspace{2pt}
\begin{tabular}{@{}r@{\hskip 0.5em}l}
Anglais: &Capacité professionnelle\\
Français: &Langue maternelle
\end{tabular}
\sectionsep

\subsection{Autres compétences}
\vspace{2pt}
\vspace{\topsep} % Hacky fix for awkward extra vertical space
\begin{tightemize}
  \item Écriture de documentation multilingue avec \altpdftext{\MyLaTeX}{LaTeX} et Markdown
  \item Activement engagé dans l'Open-Source et ses communautés
  \item Gestion de projet
\end{tightemize}


%%%%%%%%%%%%%%%%%%%%%%%%%%%%%%%%%%%%%%
%
%     COLUMN TWO
%
%%%%%%%%%%%%%%%%%%%%%%%%%%%%%%%%%%%%%%
\end{minipage}%
\hfill%
\begin{minipage}[t]{0.66\textwidth}

%%%%%%%%%%%%%%%%%%%%%%%%%%%%%%%%%%%%%%
%     EXPERIENCE
%%%%%%%%%%%%%%%%%%%%%%%%%%%%%%%%%%%%%%

\section{Expérience}
\runsubsection{La Belle Assiette}%
\descript{Software Engineer (Stage)}%
\location{sept. 2017 - août 2018 | Paris, France \scalebox{0.75}{\textit{(temps partiel, 4j/sem)}}}
La Belle Assiette amène des chefs talentueux dans votre cuisine pour que vous puissiez profiter d'un repas sans le stress de cuisiner ou de nettoyer. \href{https://labelleassiette.co.uk}{Website}
\vspace{1.5\topsep} % Hacky fix for awkward extra vertical space
\begin{tightemize}
\item Maintainance et ajout de nouvelles fonctionnalités à la plateforme de réservation, l'interface des chefs ainsi qu'à l'interface d'administration
\end{tightemize}
\sectionsep

\runsubsection{Freelancer}%
\descript{Software Engineer}%
\location{mars 2015 - présent | Lyon, France et télétravail}
\vspace{1.5\topsep} % Hacky fix for awkward extra vertical space
\begin{tightemize}
\item Création d'un programme pour télécharger automatiqument des factures PDF depuis plusieurs fournisseurs d'énergie et extraction d'informations afin d'en recueillir leur contenu
\item Création d'une solution logicielle pour automatiser la tâche répétitive de création de dossiers de remboursement dans le secteur de l'énergie verte
\item Gain d'expérience dans la gestion de projet et la gestion de demandes clientes
\end{tightemize}
\sectionsep

\runsubsection{Kids-OK}%
\descript{Software Engineer, CTO}%
\location{août 2015 – juil. 2016 | Lyon, France \scalebox{0.75}{\textit{(temps partiel, 3j/sem, then CTO full-time for 3 months)}}}
Kids-OK était une startup basée sur Lyon qui voulait offrir aux parents des solutions pour proteger leurs enfants sur les nouvelles technologies et les réseaux sociaux. \href{https://i.imgur.com/ZAXWuZN.png}{Site web (archivé)}
\vspace{\topsep}
\begin{tightemize}
\item Conception et maintenance de l'API back-end (interaction avec les APIs sociales de : Facebook, Twitter, Instagram), mise en place de tests automatisés, prototypage d'une solution de détection d'injures dans du texte
\item Développement d'une application Android : un lanceur d'applications sécurisé
\item Gain d'expérience en gestion de projet dû à la prise de lead sur la stack technique
\end{tightemize}
\sectionsep

\runsubsection{EPITECH}%
\descript{Assistant pédagogique}%
\location{févr. 2015 – juil. 2015 | Lyon, France \scalebox{0.75}{\textit{(temps partiel, 1j/sem)}}}
\vspace{\topsep}
\begin{tightemize}
\item Scriptage de tâches de vérifications et de notation roboratives
\end{tightemize}
\sectionsep

\runsubsection{AnyFetch}%
\descript{Software Engineer (Stage)}%
\location{juil. 2014 – mars 2015 | Lyon, France}
AnyFetch était une startup voulant fournir une solution de recherche de documents à travers le cloud (Gmail, Dropbox, etc.) \href{https://i.imgur.com/B3MMD6Y.png}{Site web (archivé)}, \href{https://www.producthunt.com/posts/anyfetch}{Post Product Hunt}
\vspace{\topsep}
\begin{tightemize}
\item Développement d'une extension Google Chrome en JavaScript. \href{https://i.imgur.com/yql5v1z.png}{Screenshots}
\item Maintainance d'une REST API à grande échelle (150+ appels par seconde) écrite en Node.js, plusieurs micro-services
\end{tightemize}

%%%%%%%%%%%%%%%%%%%%%%%%%%%%%%%%%%%%%%
%     Projects
%%%%%%%%%%%%%%%%%%%%%%%%%%%%%%%%%%%%%%

\section{Projets}
\runsubsection{KISS Launcher}%
\descript{Contributeur and co-mainteneur}%
\location{janv. 2015 – présent}
KISS Launcher est un lanceur d'applications ultra-rapide et open-source pour Android. \href{http://kisslauncher.com/}{Site web}
\sectionsep

%%%%%%%%%%%%%%%%%%%%%%%%%%%%%%%%%%%%%%
%     AWARDS
%%%%%%%%%%%%%%%%%%%%%%%%%%%%%%%%%%%%%%

\section{Distinctions \& Hackathons}
\begin{tabular}{@{}rll}
2016       & Finaliste   & Google Hash Code 2016\\
2015       & Participant & Wikimedia Hackathon à Lyon
\end{tabular}
\sectionsep

\end{minipage}
\end{document}
